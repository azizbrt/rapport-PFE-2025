\chapter{RELEASE 1}
\addcontentsline{toc}{chapter}{Chapitre 3 : RELEASE 1}

 

 \addcontentsline{toc}{section}{Introduction} % Ajoute "Introduction" dans la table des matières

\section*{Introduction}\\
Après avoir exposé en détail les exigences de notre projet à travers un Backlog du produit, nous commencerons dans le chapitre actuel la première version qui comprendra deux sprints, le sprint 1 et le sprint 2. Chaque sprint couvre l’analyse, la conception et la réalisation.
\section{Organisation du sprint}
Notre release est par deux Sprints :\\
\textbf{-Sprint 1 :} Configuration initiale et base du projet .\\
\textbf{-Sprint 2 :} Authentification et gestion des utilisateurs .
\subsection{Sprint 1 : Configuration initiale et base du projet .}
\subsubsection{Sprint goal}
  L'objectif de ce sprint est de mettre en place de l'environnement MERN, configuration backend et front-end et les tâches principales sont :
l'initialisation du dépôt Git
et la création du serveur Express avec une route test
aussi la modélisation de la base de données MongoDB
 et Configuration de React.js et des routes principales.\\
  \subsection{ Sprint 2 : Authentification et gestion des utilisateurs .}
  \subsubsection{Sprint goal}
  L'objectif de ce sprint est de système d'authentification (JWT) et gestion des rôles.  Tâches principales sont la création des modèles utilisateurs avec rôles et 
l'implémentation des routes d'authentification et la gestion des sessions avec JWT
Ainsi que l'interface React pour l'inscription et la connexion.
\subsubsection{Sprint backlog }
La deuxieme sprint se prolonge du 12 Février jusqu'au 21 Février . Le tableau ci-dessous représente le backlog de notre deuxieme sprint . 

