\chapter{Authentification et gestion des utilisateurs.}
\addcontentsline{toc}{chapter}{Sprint 2: Authentification et gestion des utilisateurs. }


\section*{Introduction}
\addcontentsline{toc}{section}{Introduction}
Après avoir exposé en détail les exigences de notre projet à travers un backlog produit, nous entamons dans ce chapitre la première version du projet, qui comprend deux sprints : le sprint 1 et le sprint 2. Chaque sprint couvre l’analyse, la conception et la réalisation.

\section{Organisation du sprint}
Notre release est composée de deux sprints :\\
\textbf{- Sprint 1 :} Configuration initiale et base du projet.\\
\textbf{- Sprint 2 :} Authentification et gestion des utilisateurs.

\section{Sprint 1 : Configuration initiale et base du projet}
\subsection{Objectif du sprint}
L'objectif de ce sprint est de mettre en place l’environnement MERN, la configuration backend et frontend. Les tâches principales sont :
\begin{itemize}
    \item l'initialisation du dépôt Git ;
    \item la création du serveur Express avec une route de test ;
    \item la modélisation de la base de données MongoDB ;
    \item la configuration de React.js et des routes principales.
\end{itemize}

\section{Sprint 2 : Authentification et gestion des utilisateurs}
\subsection{Objectif du sprint}
L'objectif de ce sprint est de mettre en place un système d'authentification (JWT) et une gestion des rôles. Les tâches principales sont :
\begin{itemize}
    \item la création des modèles utilisateurs avec rôles ;
    \item l'implémentation des routes d’authentification et la gestion des sessions avec JWT ;
    \item l’interface React pour l’inscription et la connexion.
\end{itemize}
\subsection{Sprint backlog}
Le 2éme sprint s’étend du 12 février au 21 février. Le tableau suivant représente le backlog de ce sprint: 


\clearpage
\renewcommand{\arraystretch}{1.6}
\setlength{\tabcolsep}{5pt}
\begin{longtable}{|c|c|m{7cm}|c|c|}
\hline
\textbf{ID} & \textbf{Sprint} & \textbf{User Story} & \textbf{Priorité} & \textbf{Complexité} \\
\hline
\endfirsthead

\hline
\textbf{ID} & \textbf{Sprint} & \textbf{User Story} & \textbf{Priorité} & \textbf{Complexité} \\
\hline
\endhead

\hline
\endfoot

\hline
\endlastfoot

\multirow{11}{*}{2} & \multirow{11}{*}{\parbox{3cm}{\centering Authentification et\\ gestion des utilisateurs}} 
& En tant que Participant, je souhaite créer un compte. & Élevée & Moyenne \\
\cline{3-5}
&& En tant que Participant, je souhaite m’authentifier. & Élevée & Moyenne \\
\cline{3-5}
&& En tant que Participant, je souhaite modifier mon compte. & Élevée & Moyenne \\
\cline{3-5}

&& En tant que gestionnaire, je souhaite obtenir un compte après la validation par l’ad-
administrateur & Élevée & Moyenne \\
\cline{3-5}
&& En tant que gestionnaire, je souhaite m’authentifier. & Élevée & Moyenne \\
\cline{3-5}
&& En tant que gestionnaire, je souhaite modifier mon compte. & Élevée & Moyenne \\
\cline{3-5}
&& En tant qu'administrateur, je souhaite modifier mon compte. & Élevée & Moyenne \\
\cline{3-5}
&& En tant qu’administrateur, je veux créer un utilisateur pour la gestion des utilisateurs. & Élevée & Moyenne \\
\cline{3-5}
&& En tant qu’administrateur, je veux modifier un utilisateur pour la gestion des utilisateurs. & Élevée & Moyenne \\
\cline{3-5}
&& En tant qu’administrateur, je veux supprimer un utilisateur pour la gestion des utilisateurs. & Élevée & Moyenne \\
\cline{3-5}
&& En tant qu’administrateur, je veux consulter un utilisateur pour la gestion des utilisateurs. & Élevée & Moyenne \\
\hline
\end{longtable}

\begin{table}[ht]
    \centering
    \caption{User Stories – Sprint Authentification et gestion des utilisateurs} 
\end{table}
\subsection{Implémentation du sprint 2}
\subsubsection{Spécification des besoins}
Dans cette section, nous identifions les besoins de notre deuxième sprint, à travers :
\begin{itemize}
    \item les diagrammes de cas d’utilisation,
    \item les descriptions textuelles associées,
    \item les diagrammes de séquences système.
\end{itemize}

\textbf {La figure ci-dessous représente le premier diagramme de cas d’utilisation de ce sprint.}

\begin{figure}[H]
    \centering
    \includegraphics[width=0.6\linewidth]{projet/images/Participant use Case.png}
    \caption{Diagramme des cas d’utilisation « S’authentifier (Participant) »}
    \label{fig:equipe_scrum}
\end{figure}

Ce diagramme décrit le processus d’authentification de l’utilisate. Nous détaillons ci-après ce cas d’utilisation sous forme textuelle :

\begin{longtable}{|>{\bfseries}p{4cm}|p{10cm}|}
\hline
Cas d’utilisation &  S'inscrire  \\
\hline
Acteurs & Participant \\
\hline
Précondition & Le participant demande l’interface d’inscription\\
\hline
Post-condition & Le compte est créé\\
\hline
Scénario principal & 
\begin{enumerate}
  \item  Le système affiche la page d’inscription.
de connexion

  \item Le participant remplit le formulaire et le soumet.

  \item Le système vérifie les informations et crée le compte
  \item Le système affiche un message de succès
\end{enumerate} 
\hline
Scénario alternatif & Le système affiche un message d’erreur si l'email déja utilisé .
 \hline
\caption{Description textuelle du cas d’utilisation pour Créer un compte}
\end{longtable}


\begin{longtable}{|>{\bfseries}p{4cm}|p{10cm}|}
\hline
Cas d’utilisation &  S’authentifier (Participant)  \\
\hline
Acteurs & Participant \\
\hline
Précondition & Le participant possède un compte\\
\hline
Post-condition & Participant Authentifié\\
\hline
Scénario principal & 
\begin{enumerate}
  \item  Le système affiche l’interface qui contient un formulaire
de connexion

  \item L’utilisateur saisit son login et son mot de passe

  \item Il confirme en cliquant sur le bouton « se connecter »
  \item Le système affiche l’interface d’accueil propre au participant
\end{enumerate} 
\hline
Scénario alternatif & Le système affiche un message d’erreur si le login ou mot de
passe sont incorrects .
 \hline
\caption{Description textuelle du cas d’utilisation « S’authentifier (Participant)  »}

\end{longtable}
\begin{longtable}{|>{\bfseries}p{4cm}|p{10cm}|}
\hline
Cas d’utilisation &  Mettre a jour de profile  \\
\hline
Acteurs & Participant \\
\hline
Précondition & Authentification préalable\\
\hline
Post-condition & les information de profile sont modifier \\
\hline
Scénario principal & 
\begin{enumerate}
  \item Le participant accède à la section « Mon profil »

  \item Le système affiche le formulaire de modification.

  \item le participant modifie les informations souhaitées et soumet le formulaire.
  \item Il clique sur « Mettre à jour ».
  \item Le système valide les données et met à jour le profil.
  \item Le système affiche un message de confirmation.
\end{enumerate} 
\hline
Scénario alternatif & Le participant  soumet le formulaire avec des champs vides ou incorrects.
 \hline
\caption{Description textuelle du cas d’utilisation « Mettre a jour d'un profile (Participant)  »}
\end{longtable}
\begin{center}
\textit{Remarque : L'étape décrit dans ce cas d’utilisation pour mettre à jour d'un profil sont également valables  aussi pour les rôles Gestionnaire et Administrateur.}
\end{center}


\textbf {La figure ci-dessous représente le deuxieme  diagramme de cas d’utilisation de ce sprint.}\\

\begin{figure}[H]
    \centering
    \includegraphics[width=0.6\linewidth]{projet/images/Gestionnaire.png}
    \caption{Diagramme des cas d’utilisation « S’authentifier (Gestionnaire) »}
    \label{fig:equipe_scrum}
\end{figure}

Ce diagramme décrit le processus d’authentification de gestionnaire. Nous détaillons ci-après ce cas d’utilisation sous forme textuelle :
\

\begin{longtable}{|>{\bfseries}p{4cm}|p{10cm}|}
\hline
Cas d’utilisation &  S’authentifier (gestionnaire)  \\
\hline
Acteurs & Gestionnaire \\
\hline
Précondition & Le gestionnaire possède un login et un mot de passe\\
\hline
Post-condition & Gestionnaire Authentifié\\
\hline
Scénario principal & 
\begin{enumerate}
  \item  Le système affiche l’interface qui contient un formulaire
de connexion

  \item Le gestionnaire saisit son login et son mot de passe

  \item Il confirme en cliquant sur le bouton « se connecter »
  \item Le système affiche l’interface d’accueil propre de gestionnaire
\end{enumerate} 
\hline
Scénario alternatif & Le système affiche un message d’erreur si le login ou mot de
passe sont incorrects .
 \hline
\caption{Description textuelle du cas d’utilisation « S’authentifier (Gestionnaire)  »}
\end{longtable}


\textbf {La figure ci-dessous représente le troisieme  diagramme de cas d’utilisation de ce sprint.}

\begin{figure}[H]
    \centering
    \includegraphics[width=0.6\linewidth]{projet/images/admin.png}
    \caption{Diagramme des cas d’utilisation « gestion d'utilisation »}
    \label{fig:equipe_scrum}
\end{figure}

Ce diagramme de cas d’utilisation représente le processus de gestion d'utilisateur ,qui se manifeste dans l’ajout, la modification, la suppression, la consultation de la liste d'utilisateur, Nous détaillons ci-après ce cas d’utilisation sous forme textuelle :

\begin{longtable}{|>{\bfseries}p{4cm}|p{10cm}|}
\hline
Cas d’utilisation &   Créer un utilisateur \\
\hline
Acteurs & Administrateur \\
\hline
Précondition & Authentification préalable\\
\hline
Post-condition & Utilisateur est ajouté\\
\hline
Scénario principal & 
\begin{enumerate}
  \item  L’administrateur clique sur le bouton d’ajout d’un utilisateur

  \item Le système affiche le formulaire d’ajout d’un utilisateur

  \item . L’administrateur remplit le formulaire avec les informations du l'utilisateur et le soumet 
  \item Le système vérifie les informations et ajoute l'utilisateur
  \item Le système affiche un message de succès
\end{enumerate} 
\hline
Scénario alternatif & 
\begin {enumerate}
\item L’administrateur soumet le formulaire avec des informations incomplètes ou incorrectes
\item L’administrateur soumet le formulaire avec les informations d’utilisateur existant
\end{enumerate} 
 \hline
\caption{Description textuelle du cas d’utilisation pour créer un utilisateur }
\end{longtable}


\begin{longtable}{|>{\bfseries}p{4cm}|p{10cm}|}
\hline
Cas d’utilisation &   Modifier un utilisateur \\
\hline
Acteurs & Administrateur \\
\hline
Précondition & Authentification préalable\\
\hline
Post-condition & Utilisateur est modifier\\
\hline
Scénario principal & 
\begin{enumerate}
  \item  L’administrateur choisit un utilisateur modifier et clique sur son bouton de modification.
  \item Le système affiche le formulaire de modification d’un utilisateur 
  \item L’administrateur modifie les informations d'utilisateur et le soumet 
  \item Le système vérifie les information setmet à jour les données du l'utilisateur
  \item Le système affiche un message de succès
  \end{enumerate} 
   \hline
   Scénario alternatif & 
   \begin {enumerate}
\item L’administrateur soumet le formulaire avec des informations incomplètes ou incorrectes
\item Le système affiche un message d’erreur
\end{enumerate} 
 \hline
\caption{Description textuelle du cas d’utilisation pour modifier un utilisateur }
\end{longtable}

\clearpage

\begin{longtable}{|>{\bfseries}p{4cm}|p{10cm}|}
\hline
Cas d’utilisation &   Supprimer un utilisateur \\
\hline
Acteurs & Administrateur \\
\hline
Précondition & Authentification préalable\\
\hline
Post-condition & Utilisateur est supprimé\\
\hline
Scénario principal & 
\begin{enumerate}
  \item  L’administrateur choisit un utilisateur  et clique sur son bouton de supprission.
  \item Le système demande une confirmation de la suppression
  \item L’administrateur confirme la suppression 
  \item Le système supprime l'utilisateur
  \item Le système affiche un message de succès
  \end{enumerate} 
   \hline
   Scénario alternatif & L’administrateur annule la confirmation
  
 \hline
\caption{Description textuelle du cas d’utilisation pour supprimer un utilisateur }
\end{longtable} \\

\renewcommand{\arraystretch}{1.4} % Réduit un peu l'espacement vertical
\begin{longtable}{|>{\bfseries}p{3.5cm}|p{9cm}|}
\hline
Cas d’utilisation & Consulter un utilisateur \\
\hline
Acteurs & Administrateur \\
\hline
Précondition & Authentification préalable \\
\hline
Post-condition & La liste des utilisateurs est affichée \\
\hline
Scénario principal & 
\begin{enumerate}
  \item L’administrateur accède à la page de gestion des utilisateurs
  \item Le système affiche un tableau contenant tous les utilisateurs
\end{enumerate} 
\hline
Scénario alternatif & Néant \\
\hline
\caption{Description textuelle du cas d’utilisation pour consulter la liste des utilisateurs}
\end{longtable}



\textbf{Les figures ci-après illustrent les diagrammes de séquences systèmes des différents cas d’utilisation de notre 2éme sprint.}

\begin{figure}[H]
    \centering
    \includegraphics[width=1\linewidth]{projet/images/Sprint 2/Participant Register.png}
    \caption{Diagramme de séquence - la création d'un compte pour le Participant}
    \label{fig:equipe_scrum}
\end{figure}
\begin{figure}[H]
    \centering
    \includegraphics[width=1\linewidth]{projet/images/Sprint 2/registration gestionnaire.png}
    \caption{Diagramme de séquence - creation de compte pour Gestionnaire }
    \label{fig:equipe_scrum}
\end{figure}

\begin{figure}[H]
    \centering
    \includegraphics[width=1\linewidth]{projet/images/Sprint 2/Authentification.png}
    \caption{Diagramme de séquence « Authentification »}
    \label{fig:equipe_scrum}
\end{figure}
\clearpage
\begin{figure}[H]
    \centering
    \includegraphics[width=1\linewidth]{projet/images/Sprint 2/diagram sequance without waterMark/cree utilisateur admin sequance diagram.png}
    \caption{Diagramme de séquence système Créer un utilisateur}
    \label{fig:diagramme2}
\end{figure}
\begin{figure}[H]
    \centering
    \includegraphics[width=1\linewidth]{projet/images/Sprint 2/diagram sequance without waterMark/consulter les utilisateurs admin sequnace diagram.png}
    \caption{Diagramme de séquence système Consulter un utilisateur}
    \label{fig:diagramme3}
\end{figure}
\begin{figure}[H]
    \centering
    \includegraphics[width=1\linewidth]{projet/images/Sprint 2/diagram sequance without waterMark/modifier utlisateur admin sequance diagram.png}
    \caption{Diagramme de séquence système Modifier un utilisateur}
    \label{fig:diagramme4}
\end{figure}
\begin{figure}[H]
    \centering
    \includegraphics[width=1\linewidth]{projet/images/Sprint 2/diagram sequance without waterMark/supprimer utilisateur admin sequance diagram.png}
    \caption{Diagramme de séquence système Supprimer un utilisateur}
    \label{fig:diagramme5}
\end{figure}
\textbf{diagramme de classe (à completer)}
\subsubsection{Réalisation}
Cette interface représente le formulaire d’inscription dédié a l'utilisateur et gestionnaire , comprenant les champs nom mail mot de passe 
\begin{figure}[H]
    \centering
    \includegraphics[width=1\linewidth]{projet/images/registration.png}
    \caption{Interface graphique de creation compte}
    \label{fig:diagramme5}
\end{figure}
\begin{figure}[H]
    \centering
    \includegraphics[width=1\linewidth]{projet/images/login.png}
    \caption{Interface graphique d’authentification}
    \label{fig:diagramme5}
\end{figure}
\clearpage
Cette interface représente la page d’accueil de l’administrateur. Elle s’affiche une fois la gérer les utilisateurs
\begin{figure}[H]
    \centering
    \includegraphics[width=1\linewidth]{projet/images/gestion d'utilisateur.png}
    \caption{Interface graphique de gérer les utilisateurs }
    \label{fig:diagramme5}
\end{figure}


