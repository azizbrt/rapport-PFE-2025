\chapter{Inscription et interaction utilisateur   }
\addcontentsline{toc}{chapter}{Chapitre 5 : Inscription et interaction utilisateur }

\section*{Introduction}
\addcontentsline{toc}{section}{Introduction}
Dans ce chapitre, nous allons étudier en profondeur la quatrième sprint de notre projet. 
\section{Organisation du sprint}
Notre chapitre est composée par :\\
\textbf{Sprint 4 :} Inscription et interaction utilisateur
\section{Sprint 4 : Inscription et interaction utilisateur}
\subsection{Objectif du sprint}\\
L'objectif de ce sprint est l'inscription aux événements et feedback utilisateur, et après tout ça le participant pourra consulter les événements.

\subsection{Sprint backlog}
Le 4éme sprint s’étend du  05 mars au 14 mars. Le tableau suivant représente le backlog de ce sprint:
\begin{table}[h!]
\renewcommand{\arraystretch}{1.6}
\setlength{\tabcolsep}{5pt}
\centering
\begin{tabular}{|c|c|m{7cm}|c|c|}
\hline
\textbf{ID} & \textbf{Sprint} & \textbf{User Story} & \textbf{Priorité} & \textbf{Complexité} \\
\hline
\multirow{5}{*}{4} & \multirow{5}{*}{\parbox{3cm}{\centering Inscription et interaction utilisateur}}
& En tant que Participant. je veux consulter les évenement & Élevée & Faible \\
\cline{3-5}
&& En tant que Participant. je veux  Inscrire a un évenement. & Moyenne & Faible \\
\cline{3-5}
&& En tant que Participant. je veux  Annuler l'inscription "en attente" . & Moyenne & Faible \\
\cline{3-5}
&&En tant que Participant. je veux  Envoyer des messages pour demande d'information & Moyenne & Faible\\
\cline{3-5}
&& En tant que Participant , je veux faire des commentaires & Moyenne & Trés  faible \\
\hline
\end{tabular}
\end{table}
\subsection{Implémentation du sprint 4}
\subsubsection{Spécification des besoins}
Dans cette section, nous identifions les besoins de notre  sprint, à travers :
\begin{itemize}
    \item les diagrammes de cas d’utilisation,
    \item les descriptions textuelles associées,
    \item les diagrammes de séquences système.
\end{itemize}

\textbf {La figure ci-dessous représente le diagramme de cas d’utilisation de ce sprint par rapport aux participants.}
\begin{figure}[H]
    \centering
    \includegraphics[width=0.6\linewidth]{projet/images/sprint4/partisipant1.png}
    \caption{Diagramme des cas d’utilisation d'inscription et interaction utilisateur }
    \label{fig:Adminee }
\end{figure}
\clearpage
Ce diagramme décrit le processus d'inscription et interaction utilisateur pour le Participant. Nous détaillons ci-après ce cas d’utilisation sous forme textuelle :

\begin{longtable}{|>{\bfseries}p{4cm}|p{10cm}|}
\hline
Cas d’utilisation & Consutler un événement \\
\hline
Acteur & Participant \\
\hline
Précondition & Le Participant est authentifié \\
\hline
Post-condition & Les événements sont affichés \\
\hline
Scénario principal & 
\begin{enumerate}
  \item le Participant accède à la page d’accueil ou de recherche.
  \item  Le système affiche la liste des événements disponibles
\end{enumerate} 
\hline
Scénario alternatif &  Aucun événement n’est disponible : un message “Aucun événement disponible” est affiché.
\hline
\caption{Description textuelle du cas d’utilisation pour consutler  un événement}
\end{longtable}


\begin{longtable}{|>{\bfseries}p{4cm}|p{10cm}|}
\hline
Cas d’utilisation & S'inscrire à un événement \\
\hline
Acteur & Participant \\
\hline
Précondition & Être connecté et consulter un événement\\
\hline
Post-condition & Le participant est inscrit à l’événement \\
\hline
Scénario principal & 
\begin{enumerate}
  \item le Participant choisit un événement.
  \item   Il clique sur "S'inscrire maintenant".
  \item Le système affiche les détails du paiement 
  \item le participant  saisit les informations nécessaires et clique sur  "Procéder au paiement".
\item Le système affiche une nouvelle page contenant :(Le prix,RIB )
\item Le participant clique sur button "Valider le paiement".
\item Le système enregistre l’intention de paiement et affiche un message de confirmation .

\end{enumerate} 
\hline
Scénario alternatif &  Le participant ferme la page avant de valider le paiement

\hline
\caption{Description textuelle du cas d’utilisation pour inscrire à un événement}
\end{longtable}
\clearpage
\begin{longtable}{|>{\bfseries}p{4cm}|p{10cm}|}
\hline
Cas d’utilisation & Annuler l'inscription "en attente" \\
\hline
Acteur & Participant \\
\hline
Précondition &
\begin{enumerate}
  \item Être connecté .
  \item L'inscription a le statut "en attente"
  \end{enumerate} 
\hline
Post-condition & 
\begin{enumerate}
    \item Si l'inscription est encore en attente, elle est annulée avec succès (supprimée ou marquée comme "annulée").
    \item Si l'inscription est déjà validée, on ne peut pas faire l'annulation
\end{enumerate}
\hline
Scénario principal & 
\begin{enumerate}
  \item le Participant Accède à "Mon profil".
  \item Le système affiche les inscription en attente 
  \item le participant sélectionne une inscription avec le statut “en attente” et Il clique sur “Annuler”
\item Le système affiche un message "Êtes-vous sûr de vouloir annuler cette inscription ?" 
\item  Le Participant clique sur “OK”
\item  Le système affiche le message : “Inscription supprimée avec succès”

\end{enumerate} 
\hline
Scénario alternatif & Le Participant clique sur “Non” dans la boîte de confirmation : aucune action n’est effectuée.
\hline
\caption{Description textuelle du cas d’utilisation pour annuler l’inscription "en attente"}
\end{longtable}

\begin{longtable}{|>{\bfseries}p{4cm}|p{10cm}|}
\hline
Cas d’utilisation & Envoyer des messages pour demande d'information \\
\hline
Acteur & Participant \\
\hline
Précondition & Le participant est connecté au site.\\
\hline
Post-condition & La demande est envoyée à l'admin\\
\hline
Scénario principal & 
\begin{enumerate}
  \item Le participant accéde à la page de contact.
  \item Le participant  saisit son adresse e-mail dans le champ prévu.et rédige son message et clique sur "Envoyer"
  \item Le système affiche un message de confirmation 
  (« Message envoyé avec succès »).
  \item Le message est envoyé à l’administrateur.
\end{enumerate} 
\hline
Scénario alternatif & Le système affiche un message d’erreur : « Veuillez remplir tous les champs correctement. »
\hline
\caption{Description textuelle du cas d’utilisation pour Envoyer des messages pour demande d’information .}
\end{longtable}


\begin{longtable}{|>{\bfseries}p{4cm}|p{10cm}|}
\hline
Cas d’utilisation & Faire un commentaire \\
\hline
Acteur & Participant \\
\hline
Précondition & 
\begin{enumerate}
  \item Le participant est connecté au site.
  \item L’événement n’est pas encore terminé.
\end{enumerate} 
\hline
Post-condition & Le commentaire est visible par les autres \\
\hline
Scénario principal & 
\begin{enumerate}
  \item Le participant accède à la page d’un événement.
  \item Le système vérifie que l’événement est encore actif.
  \item Dans la section "Laissez un commentaire", il écrit un message et clique sur le bouton Publier.
  \item Le système enregistre et affiche le commentaire. 
  
\end{enumerate} \\
\hline

Scénario alternatif & 
\begin{enumerate}
  \item Si le participant non connecté tente d'accéder à la section commentaire,
  \item Le champ de saisie et le bouton Publier sont désactivés ou masqués.
\end{enumerate} 
\hline
\caption{Description textuelle du cas d’utilisation pour Faire un commentaire.}
\end{longtable}


