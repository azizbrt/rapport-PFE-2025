\chapter*{Introduction Générale}
%ajouter le titre introduction générale dans la table des matières
\addcontentsline{toc}{chapter}{Introduction Générale}
%définir l'entête de ce chapitre 
\markboth{Introduction Générale}{}


Les nouvelles technologies de l’information sont en train de modifer fondamentalement la
manière dont les entreprises sont gérées, et en particulier la manière dont elles recherchent,
traitent, échangent et dffusent de l’information.
Dans ce contexte , la société OMINET  s’est proposée de développer une Plateforme de Gestion d’Événements Communautaires.
 offrira aux utilisateurs qu’ils soient  participants, gestionnaires d’événements ou administrateurs une interface moderne e pour interagir efficacement.

Les gestionnaires pourront consulter les événements disponibles, et les participant s’inscrire, réserver des places en ligne, donner leur avis et échanger avec une communauté dynamique. Les administrateurs auront accès à des outils avancés pour organiser et gérer les événements, superviser les inscriptions et analyser les retours des participants, ce qui garantira une meilleure gestion et une expérience utilisateur optimisée.

Le premier chapitre intitulé « Présentation du cadre du projet» est consacré à la présentation de
l’organisme d’accueil qui est la OMINET et plus précisemment le département d’accueil ,ainsi qu’une
vision détaillée sur le cadre du projet tout en incluant le contexte du projet puis une petite
critique sur l’existant pour pouvoir à la suite présenter la solution .  Par la suite , nous allons
mettre en évidence la méthodologie suivie durant la réalisation de ce projet . 

Dans le chapitre suivant, intitulé « Planification du projet  » nous analysons en détail les besoins
fonctionnels et non fonctionnels ainsi que la specification de ces besoins à travers les diagrammes
de cas d’utilisation et l'environnement de développement et choix techniques.  De plus, nous exposerons le product backlog
et la planification des sprints pour avoir une vision claire du projet.

Ensuite, Les quatres derniers chapitres seront consacrés aux Sprints de notre projet
qui vont dévoiler chacun les étapes suivies durant la réalisation des différentes sessions de notre
applicaion tout en introduisant leurs fonctionnalités .
