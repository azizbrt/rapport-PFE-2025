\chapter*{Introduction Générale}
%ajouter le titre introduction générale dans la table des matières
\addcontentsline{toc}{chapter}{Introduction Générale}
%définir l'entête de ce chapitre 
\markboth{Introduction Générale}{}


Depuis l’émergence de l’informatique, l’homme n’a cessé de développer des langages et des outils pour concevoir le web, mêlant créativité et logique afin de créer des expériences en ligne interactives et fonctionnelles. Aujourd’hui, l’informatique est devenue un pilier central dans la gestion de l’information et un atout majeur dans le monde professionnel en perpétuelle évolution.

Pour les étudiants, acquérir de l’expérience dès leurs études est crucial. Pourtant, la recherche de stages reste un véritable défi : trouver des opportunités alignées avec leurs compétences et aspirations, tout en naviguant dans un écosystème professionnel souvent complexe.

C’est dans cette optique que notre projet prend tout son sens. Nous proposons la création d’une plateforme innovante dédiée à la gestion d’événements. Cet outil offrira aux utilisateurs qu’ils soient simples participants, gestionnaires d’événements ou administrateurs une interface moderne et intuitive pour interagir efficacement.

Les utilisateurs pourront consulter les événements disponibles, s’inscrire, réserver des places en ligne, donner leur avis et échanger avec une communauté dynamique. De leur côté, les administrateurs disposeront d’outils avancés pour organiser et gérer les événements, superviser les inscriptions et analyser les retours des participants, garantissant ainsi une meilleure gestion et une expérience utilisateur optimisée.